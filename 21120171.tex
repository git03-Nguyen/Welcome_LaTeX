%Nguyen Dinh Anh - 21120171
%BTVN tuan 03 - Thuc hanh Nhap mon CNTT
%Code written with Overleaf + TexMaker  
%--------------------------------------------------------------------

%Define the document class, size,..
\documentclass[10.5pt,a4paper]{article}
%Import VNese but not for \date, \chapter,...
\usepackage[vietnamese,english]{babel}
%More needed packages
\usepackage{amsmath}
\usepackage{graphicx}

%Define title, author, date
\title{Hello Latex!}
\author{Nguyen Dinh Anh - 21120171}
\date{\today}
%---------------------------------------------------------------------

\begin{document}
    \maketitle
    
    %SECTION 1: GETTING STARTED
    \section{Getting Started}
        \label{sec:getstarted}
        
        \textbf{Hello Latex!} Today I am learning \LaTeX. \LaTeX\ is a great program for writing math. I can test Vietnamese language "Học vui thôi".\\
        \indent I can write in line math such as $a^2+b^2=c^2$. I can also give equations their own space:
        \begin{equation}
            \label{eq:example}
            \gamma ^2 + \theta ^2 = \omega ^2
        \end{equation}
        \indent ``Maxwell's equations'' are named for James Clark Maxwell and are as follow:
        \begin{align}
            \label{eq:gauss1}
            &\vec{\nabla} \cdot \vec{E}=\frac{\rho}{\epsilon_0} &&&&\text{Gauss's Law} \\ 
            \label{eq:gauss2}
            &\vec{\nabla} \cdot \vec{B}=0 &&&&\text{Gauss's Law for Magnetism}
        \end{align}
        Equations \ref{eq:gauss1} and \ref{eq:gauss2} are some of the most important in Physics.
    
    %SECTIONS 2: WHAT ABOUT MATRIX EQUATIONS?
    \section{What about Matrix Equations?}
        \label{sec:matrixeq}
        
        \begin{equation*}
            \begin{pmatrix}
                a_{11} & a_{12} & \ldots &a_{1n}\\
                a_{21} & a_{22} & \ldots &a_{2n}\\
                \vdots & \vdots & \ddots & \vdots\\
                a_{n1} & a_{n2} & \ldots & a_{nn}
            \end{pmatrix}
            \begin{bmatrix}
                v_1\\
                v_2\\
                \vdots\\
                v_n
            \end{bmatrix}
                =
                \begin{matrix}
                w_1\\
                w_2\\
                \vdots\\
                w_n
            \end{matrix}
        \end{equation*}
    
    %SECTION 3: TABLES AND FIGURES
    \section{Tables and Figures}
        \label{sec:tablesfigures}
        
        Creating a Table is not unlike creating a matrix:
        \begin{table}
            \centering
            \caption{This is a table that shows how to create different lines as well as different justifications}
            \begin{tabular}{|l||c|c|r|}
                \hline
                x & 1 & 2 & 3\\
                \hline
                $f(x)$ & 4 & 8 & 12\\
                \hline
                f(x) & 4 & 8 & 123\\
                \hline
            \end{tabular}
        \end{table}
        
        \pagebreak %The end of page 1
        
        \begin{figure}
            \centering
            \includegraphics[width=1\textwidth]{img\hcmus}
            \caption{\label{fig:hcmus}HCM University of science}
        \end{figure}
    
    %REFERENCES
    \section{Bibliography}
        \label{sec:bib}
        
        You will probably want references in your document so that you can cite articles like \cite{frenkel_fine_2013,frenkel_optical_2013,frenkel_temperature_2012}
        
        \bibliographystyle{ieeetr} %Set the bib style to "ieeetr"
        \bibliography{refs\references}

\end{document}
